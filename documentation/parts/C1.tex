% !TEX root= ...main.tex

\section{Общие сведения}

\subsection{Наименование программы}

%
Полное наименование програмы: 
\begin{center}
	\textbf{Geo-Information System Excelsior}
\end{center}

Сокращенно: 
\begin{center}
	\textbf{GIS Excelsior}, \textbf{GIS} или \textbf{Exclesior}
\end{center}

Полное наименование программы на русском языке:
\begin{center}
	\textbf{Гео-Информационная система Excelsior}
\end{center}

Сокращенно: 
\begin{center}
	\textbf{ГИС Excelsior}, \textbf{ГИС} или \textbf{Excelsior}
\end{center}
%



\subsection{Необходимое программное обеспечение}

%
Для полной работоспособности программы хватит обычного браузера, которым вы пользуетесь ежедневно. Наша программа отлично работает с популярными браузерами, такими как Google Chrome, Firefox, Microsoft Edge и др. Для работы с картой желательно иметь компьютерную мышь, если вы используете ПК.

К сожалению, наше ПО еще не вышло в сеть и не выложено ни на одном хостинге, поэтому развертывание только локальное, именно поэтому вы скорее всего не сможете использовать свой телефон или планшет для работы с нашей системой. 

\underline{НО!} Наша программа масштабируема и вполне может быть развернута на мобильных устройствах не только как сайт, но и как полноценное приложение.
%



\subsection{Развертывание GIS Excelsior}

%
В этом параграфе мы рассмотрим как развернуть на своем ПК нашу программу. Первостепенно вам необходимо перейти по ссылке на ресурс, чтобы скачать GIS. Итак:

\begin{enumerate}
	\item Перейти по \href{https://github.com/temikfart/GIS-Excelsior/archive/refs/heads/master.zip}{ссылке} для скачивания архива формата <<.zip>>. Эта ссылка ведет на ресурс \textbf{GitHub}, на котором и расположен \href{https://github.com/temikfart/GIS-Excelsior/tree/master}{исходный код проекта};
	
	\textit{Примечание:} На момент написания этой инструкции наш исходный код находится в открытом доступе.
	\item Используя стороннее или встроенное ПО, вам необходимо разархивировать скачанный файл в удобное для вас на компьютере место;
	\item Перейти в директорию с нашей программой и перейти в папку \textbf{system}. Здесь вам необходимо запустить файл \textbf{index.html}.
\end{enumerate}

После последнего шага вы должны будете увидеть как наша программа запущена в вашем браузере по умолчанию. Она готова к использованию!
%



%\subsection{Используемые языки программирования}
%
%%
%Наша программа написана на трёх языках программирования: \textbf{JavaScript}, \textbf{HTML} и, конечно же, \textbf{CSS}. В большей степени (порядка 95\%) код написан на языке \textbf{JavaScript}. {\Large ALERT!} \textit{Описание какие элементы программы на каком языке написаны и почему.}
%%
